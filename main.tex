\documentclass[fleqn, titlepage, 12pt]{report}
\author{Atanas Ormanov}
\usepackage[T2A]{fontenc}
\usepackage[utf8]{inputenc}
\usepackage[bulgarian,english]{babel}
\usepackage{amsmath,amssymb}
\usepackage{geometry}
\geometry{margin=0.8in}

\newcommand*\rightbijectionarrow{\mathrel{\ooalign{$\twoheadrightarrow$\cr\hidewidth\hbox{$\rightarrowtail\mkern 1mu$}\cr}}}
\newcommand*\rightcircarrow{\mathrel{\ooalign{$\rightarrow$\cr\hidewidth\hbox{$\circ\mkern 5mu$}\cr}}}

\title{Записки по Теория на Множествата \\
\large При проф. Тинко Тинчев}

\begin{document}
\parindent 0pt
\maketitle

\begin{center}
  \underline{\huge\normalfont Лекция 1}
\end{center}
\bigbreak

Book recomendation:\\
Introduction to Set Theory (3d Edition) by Karel Hrbackeck \& Toomas Yech
\bigbreak

Съпоставка м-ду Актуална безкрайност и Потенциална безкрайност:\\
На пръв поглед ако $B \subseteq A$ и $B \neq A$, то B има по-малко елементи, но при безкрайни мн-ва не е задължително.
\bigbreak

\fbox{def} Принцип за неограничената абстракция:\\
Нека $\mathcal{A} (x)$ е едноместно свойство на обекта $x$. Тогава има множество A, такова че
$x \in A \Leftrightarrow \mathcal{A} (x)$
\bigbreak

Парадокс на Ръсел: \\
Нека $\mathcal{R}$ е св-во такова че $\mathcal{R} (x) \Leftrightarrow x \notin x$ за произволно $x$
\bigbreak

От принципа за неограничената абстракция (*) - има множество R, \\
такова че $x \in R \Leftrightarrow \mathcal{R} (x)$ за произволно x \\
... $R \in R \Leftrightarrow R \notin R$
\bigbreak

Езикът на теория на множествата се състои от:
\begin{itemize}
\item Двуместни свойства: =, $\in$ \\
(равенство в смисъла на Лайбниц означава неотличимост)
\item Булеви връзки: $\lor, \land, \lnot, \Rightarrow, \Leftrightarrow$
\item Квантори: $\forall{x}\phi, \exists{x}\phi$
\end{itemize}
ZF - аксиоми на Цермело Френкел \\
ZFC - ZF заедно с аксиомата за избора
\bigbreak

\fbox{def} Теоритико множествени свойства: \\
В света (универсума) има само множества (това са обектите с които ще работим)
\bigbreak

ТМ свойствата разделяме на: \\
1) Логически аксиоми
\begin{itemize}
  \item $\forall{x}\forall{y}(x = y \Rightarrow y = x)$
  \item $\forall x \forall y (x = y \Rightarrow y = x)$
  \item $\forall{x}\forall{y}\forall{z}(x = y \land y = z \Rightarrow x = z)$
  \item $\forall{x}\forall{y}\forall{z}(x \in y \land y = z \Rightarrow x \in z)$
  \item $\forall{x}\forall{y}\forall{z}(x = y \land y \in z \Rightarrow x \in z)$
\end{itemize}
1-3 са аксиомите за еквивалентност на равенството\\
4-5 са аксиомите за конгруентност
\bigbreak

2) Аксиоми на ZF:
\begin{enumerate}
  \item $\exists{x}(x = x)$  - Има поне един обект в света
  \item $\forall{x}\forall{y}(\forall{z}(z \in x \Leftrightarrow z \in y) \Rightarrow x = y))$
- Обемност / екстенсионалност. Ако 2 множества имат едни и същи елементи, то те са равни.
  \item $\exists{x}(x = x)$ - принцип за ограничената абстракция / схема \\
    за отделянето.
\end{enumerate}
\underline{Док 2.2:} $\forall{y}\forall{z}(y = z \Rightarrow \forall{x}(x \in y \Leftrightarrow x \in z))$ \\
Нека предположим че $y = z$, нека x е произволно множество. Използваме логическа аксиома 4 за да докажем.
\bigbreak

\underline{Док 2.3:} Нека $\phi (x, u_1, u_2, ... , u_n)$ e ТМ. св-во, нека $u_1, ..., u_n$ са произволно мн-ва.
Всеки път, когато А е множество, съществува множество, чийто елементи са точно онези елементи на А, за които е в сила
$\phi (x, u_1, u_2, ..., u_n)$. \\
$\forall{u_1 u_2 ... u_n}\forall{A}\exists{B}\forall{x}(x \in B \Leftrightarrow x \in A \land \phi (x, u_1, u_2, ..., u_n))$ - св-во на x.
\bigbreak

\fbox{Тв} При фиксирани $A, u_1, ..., u_n$ - множества и теоритико множествено свойство $\phi$,\\
съществуват единствено множество B, за което
$\forall{x}(x \in B \Leftrightarrow x \in A \land \phi (x, \overline{u}))$, където $\overline{u}$ са параметри.
\bigbreak

\underline{Док:} Нека $B_1$ и $B_2$ са такива мн-ва, че:
$\forall{x}(x \in B_1 \Leftrightarrow x \in A \land \phi (x, \overline{u}))$
$\forall{x}(x \in B_2 \Leftrightarrow x \in A \land \phi (x, \overline{u}))$
Искаме да док че $B_1 = B_2$. Нека $y \in B_1$ е произволен. Тогава $y \in A \land \phi (y, \overline{u})$.
Следователно $y \in B_2$. Така $\forall{x}(x \in B_1 \Leftrightarrow x \in B_2)$. От аксиомата за обемност $B_1 = B_2$.
\bigbreak

\fbox{Тв} Съществува празно множество
\bigbreak

\underline{Док:} Ще докажем че $\exists{A}\forall{x}(x \notin A)$ \\
Нека B е множество (От. аксиомата 0). Нека $\phi (x) \leftrightharpoons \lnot (x = x)$. Нека M е единственото множество, такова че
$\forall{x}(x \in M \Leftrightarrow x \in B \land \phi (x))$. Ще док. че $\forall{x}(x \notin M)$.
Допускаме че $x \in M$ е произволно. Тогава $x \in B \land \phi (x)$. Така $\phi (x)$, т.е. $x \neq x$,
противоречи с 1-вата лог. аксиома. Следователно $x \notin M$. Понеже x е произволно то $\forall{x}(x \notin M)$.
\bigbreak

\fbox{Опр} За множество А, параметри $\overline{u}$ и св-во $\phi$, съществува единствено такова B, което бележим така:
$B = \{x\ |\ x \in A \land \phi (x, \overline{u})\}$
\bigbreak

\fbox{Тв} Съществува единствено празно множество.
\bigbreak

\underline{Док:} Нека $M_1$ и $M_2$ са празни, т.е. $\forall{x}(x \notin M_1)$ и $\forall{x}(x \notin M_2)$
Нека t е произволно множество, тогава $t \notin M_1$ и $t \notin M_2$. Но t беше произволно,
значи $t \in M_1 \Leftrightarrow t \in M_2$ и от аксиомата за обемност $M_1 = M_2$
Празното множество бележим с $\emptyset$
\bigbreak

Означение: $A \subseteq B \leftrightharpoons \forall{x}(x \in A \Rightarrow x \in B)$
\bigbreak

\fbox{Тв} За всяко множество А, е изп. че $\emptyset \subseteq A$
\bigbreak

\fbox{Тв} Не същ. множество, което съдържа всички мн-ва: $\lnot \exists{A}\forall{x}(x \in A)$
\bigbreak

\underline{Док:} Допускаме противното. Нека B е такова че $\forall{x}(x \in B)$\\
Нека $R = \{x\ |\ x \in R \land x \notin x\}$. Използваме аксиомата схема за отделяне с A = B и
$\phi (x) \leftrightharpoons x \notin x$.
Отделяме онези x, за които $x \notin x$. Така R е множество. Тогава $R \in B$.
Получаваме че $R \in R \Leftrightarrow R \in B \land R \notin R \Leftrightarrow R \notin R$ - противоречие с допускането.
Тоест няма такива мн-ва.
\bigbreak Future reading:

\begin{itemize}
  \item actual infinity vs potential infinity
  \item Banach–Tarski paradox (occurs after the patch of Russel's paradox)
  \item Cantor's definition of real numbers
\end{itemize}
\bigbreak

\clearpage
\begin{center}
  \underline{\huge\normalfont Лекция 2}
\end{center}
\bigbreak

\fbox{Тв} За всеки две множества $A$ и $B$, съществува единствено множество C,\\
такова че $ \forall{x}(x \in C \Leftrightarrow x \in A \land x \in B)$.
\bigbreak

\underline{Док за съществуване:} Нека $\phi(x,u) \leftrightharpoons x \in u$.
Според аксиомната схема за отделяне в-ху множеството $A$ и $\phi$ за $u = B$,
същ. множество $ C = \{x\ |\ x \in A \land \phi(x,B)\} = \{x\ |\ x \in A \land x \in B\}$. Значи за всяко $x$,
$x \in C \Leftrightarrow x \in A \land x \in B$
\bigbreak

\underline{Док за единственост:} Нека $C_1$ и $C_2$ са такива мн-ва, че
$x \in C_i \Leftrightarrow x \in A \land x \in B, i = 1, 2$. \\
Тогава за всяко x,
$x \in C_1 \Leftrightarrow x \in A \land x \in B \Leftrightarrow x \in C_2$ и по аксиомата за обемност $C_1 = C_2$.\\
Това множество означаваме с $A \cap B$.
\bigbreak

\fbox{Тв} За всеки две множества A и B, съществува единствено множество C,\\
такова че $ \forall{x}(x \in C \Leftrightarrow x \in A \land x \notin B)$
\bigbreak

$\phi(x,u) \leftrightharpoons x \notin u$, т.е. отделяме от А всички ел. x, за които $\phi(x,B)$\\
$x \in C \Leftrightarrow x \in A \land x \notin B, i = 1, 2$\\
$x \in C_1 \Leftrightarrow x \in C_2, \forall{x}$\\
$C_1 = C_2$\\
Това единствено множество бележим $A \setminus B$ и наричаме разлика на A и B.
\bigbreak

Можем да правим "голямо" сечение
\bigbreak

\fbox{Тв} Нека $A \neq \emptyset$. Тогава съществува единствено множество $B$,\\
което съдържа точно множествата, които са елементи на всеки един елемент на A.\\
$ \forall{x}(x \in B \Leftrightarrow \forall{y}(y \in A \Rightarrow x \in y))$
\bigbreak

\underline{Док за същ.:} Нека $y_0 \in A$, защото А е непразно. Нека
$\phi(x,u) \leftrightharpoons \forall{y}(y \in u \Rightarrow x \in y)$.
От аксиомната схема за отделянето, има множество\\
$B' = \{x \in y_0 \land \phi(x,A\} = \{x\ |\ x\in y_0 \land \forall{y}(y \in A \Rightarrow x \in y\}$

Ще док че $\forall{x}(x \in B' \Leftrightarrow \forall{y}(y \in A \Rightarrow x \in y))$
Нека $x \in B'$. Тогава $x \in y_0 \land \forall{y}(y \in A \Rightarrow x \in y)$,
в частност $ \forall{y}(y \in A \Rightarrow x \in y)$.\\
Обратното, нека x е т.ч. $ \forall{y}(y \in A  \Rightarrow x \in y)$.\\
Но $y_0 \in A$, следователно $x \in y_0$. Така $x \in y_0 \land \forall{y}(y \in A \Rightarrow x \in y)$
от където $x \in B'$
\bigbreak

\underline{Док единств.:} Нека $B_1$ и $B_2$ са такива мн-ва че ... $ x \in B_i \Leftrightarrow \forall{y}(y \in A \Rightarrow x \in y)$ за $i = 1,2$\\
Така за всяко х, $ x\ in B_1 \Leftrightarrow \forall{y}(y \in A \Rightarrow x \in y) \Leftrightarrow x \in B_2$
Т.е. има единствено такова множество, бележим го $\bigcap A$ или $\bigcap_{x \in A}x$
\bigbreak

Приемаме че $\bigcap\emptyset \leftrightharpoons \emptyset$
\bigbreak

\fbox{Аксиома за чифта} За всеки 2 мн-ва a и b, съществува множество A, измежду чиито ел. са a и b.\\
$ \forall{a} \forall{b} \exists{A} (a \in A \land b \in A)$
\bigbreak

\fbox{Тв} За всеки 2 мн-ва a и b същ. единствено множество B, т.ч. $ \forall{x}(x \in B \Leftrightarrow x = a \lor x = b)$
\bigbreak

\underline{Док ед.:} Нека $B_1$ и $B_1$ са мн-ва, т.ч. $ \forall{x}(x \in B_i \Leftrightarrow x = a \lor x = b)$\\
Тогава за всяко х, $x \in B_1 \Leftrightarrow x = a \lor x = b \Leftrightarrow x \in B_2)$
След. $B_1 = B_2$
\bigbreak

\underline{Док същ.:} Нека А е такова множество че $a \in A$ и $ b \in A$. Нека $ \phi(x, u_1, u_2) \leftrightharpoons x = u_1 \lor x = u_2))$\\
По аксиомата схема за отд., същ. множество $B = \{ x\ |\ x \in A \land \phi(x,a,b)\}$.\\
Ще док. че $ \forall{x}(x \in B \Leftrightarrow x = a \lor x = b)$. Нека x е произв. и нека $x \in B$.
Тогава $x \in A \land \phi(x,a,b)$, в частност $ \phi(x,a,b)$ т.е. $x = a \lor x = b$.\\
Нека сега $x = b \lor x = b$. Така $ \phi(x,a,b)$. Понеже $a \in A$ и $b \in A$, то $x \in A$. Следователно $x \in B$\\
Това единствено множество ще означаваме $\{a,b\}$ и ще нар. чифт на A и B.
\bigbreak

Заб: Ако $a = b$, то $ \{a,a\} = \{a\}$ наричаме синглетон на a.
\bigbreak

Определимо е в езика на ТМ дали $x$ е синглетон.\\
$x$ е синглетон
$ \Leftrightarrow \exists{a}(x = \{a\}) \Leftrightarrow \exists{a} \forall{y}(y \in x \Leftrightarrow  y = a)$.\\
Тогава можем да използваме "синглетон" като свойство във ф-ла. Сега ясно се вижда че сме разширили езика защото следните
са различни $ \emptyset, \{\emptyset\}, \{ \{\emptyset\}\}$ и т.н. (така получаваме безкрайна редица)
\bigbreak

\fbox{Св} $\{ a, b\} = \{ b, a\}$. Ясно се вижда че $ \forall{x} (x \in \{ a,b\} \Leftrightarrow x \in \{ b, a\})$
\bigbreak

\fbox{def} $<a,b> = <a_1,b_1> \Leftrightarrow a = a_1 \land b = b_1$
\bigbreak

\fbox{Опр} Наредена двойка на мн-вата x и y наричаме множеството $ \{ \{ x\}, \{ x, y\}\}$ и ще означаваме с $<x,y>$.
\bigbreak

Заб: Ако използваме x вместо $ \{ x\}$ ще можем да правим цикли на принадлежност - $A \in B \in  C$. другия път ще
въведем "правило" което ще забрани такива неща.
\bigbreak

\fbox{Тв} За всяко $x1, y1, x2, y2$ е в сила, че $<x1,y1> = <x2,y2> \Leftrightarrow x1 = x2 \land y1 = y2$
\bigbreak

\underline{Док:} $ (\Leftarrow) x1 = x2 \land  y1 = y2$,
показваме че $ \{ x1\} = \{ x2\} \land \{ x1, y1\} = \{ x2,y2\}$\\
$ \{ \{ x1\}, \{ x1, y1\}\} = \{ \{ x2\}, \{ x2, y2\}\}$ и от там $<x1,y1> = <x2,y2>$
\bigbreak

$(\Rightarrow)$ Нека $<x1,y1> = <x2,y2>$
\begin{enumerate}
  \item $x1 = y1$, тогава $<x1,y1> = \{ \{ x1\}, \{ x1,x2\}\} = \{ \{ x1\}, \{ x1\}\} = \{ \{ x1\}\} = <x2,y2> = 
    \{ \{ x2\}, \{ x2,y2\}\}$. Следователно $ \{ x1\} = \{ x2\} = \{ x2, y2\}$. Така: $x1 = x2$ и $x2 = y2$.
    Тогава $x1 = x2 = y2 = y1$
  \item $x1 \neq y1$. Тогава $ \{ x1\} \neq \{ x1, y1\}$. Тогава $ \{ x2\} \neq \{ x2, y2\}$. Тогава $y2 \neq x2$, защото
    иначе чифта и синглетона щяха да съвпадат. От тук $ \{ x1\} \neq  \{ x2, y2\}$. Но $ \{ x1\} \in <x2,y2>$, и така
    $ \{ x1\} = \{ x2\}$. След. $ \{ x1, y1\} \neq \{ x2\}$, от където $\{ x1, y1\} = \{ x2, y2\}$.
    От $ \{ x1\} = \{ x2\}$, следва че $x1 = x2$. Тогава $ \{ x1, y1\} = \{ x2, y1\} = \{ x2, y2\}$.
    Понеже $y1 \neq x1 = x2$, то $y1 = y2$
\end{enumerate}
\bigbreak

\fbox{Аксиома за обединение} За всяко множество A съществува множество B, т.ч. всеки елемент на елемент на A е елемент на B.\\
$ \forall{x} \forall{y} (x \in y \land y \in A \Rightarrow x \in B)$
\bigbreak

\fbox{Тв} За всяко множество А съществува единствено множество B,\\
т.ч. $ \forall{x} (x \in B \Leftrightarrow \exists{y} (y \in A \land x \in y))$
\bigbreak

\underline{Док за ед:}
$i = 1, 2. \forall{x} (x \in B_i \Leftrightarrow \exists{y} (y \in A \land x \in y))$ за всяко x,\\
$x \in B_1 \Leftrightarrow \exists{y} (y \in A \land x in y)\Leftrightarrow x \in B_2$, т.е. $B_1 = B_2$
\bigbreak

\underline{Док за същ.} Нека C е такова множество, че
$ \forall{x} \forall{y} (y \in A \land x \in y \Rightarrow x \in C)$.\\
Нека $B = \{x\ |\ x \in  C \land \exists{y} (y \in A \land x \in y)\}$\\
Сега ако $x \in B \implies x \in C \land \exists{y} (y \in A \land x \in y) \implies \exists{y} (y \in A \land x \in y)$\\
Нека $ \exists{y} (y \in A \land x \in y)$. Нека $y_0$ е свидетел за това $(y_0 \in A \land x \in y_0)$.\\
Понеже $y_0 \in A \land x \in y_0$, то $x \in C$. Следователно $x \in B$
\bigbreak

Значи съществува такова множество и то е единствено. Ще го бележим с $\bigcup A$.
\bigbreak

Заб: Означение означава че ще го използваме във формула като съкращение(syntax sugar).
\bigbreak

Не може да се дефинира операция за допълнение. Тоест:\\
\fbox{Тв} За нито едно множество $A$ не съществува множество
$\overline A$, т.ч. $ \forall{x} (x \in \overline{A} \Leftrightarrow x \notin A)$
\bigbreak

\underline{Док:} Допускаме противното - нека $A$ и $\overline{A}$ са такива мн-ва, такова че за всяко x
$x \in \overline{A} \Leftrightarrow x \notin A$.\\
Нека $V = \bigcup \{ A, \overline{A}\}$ - от аксиомата за чифта и обединението. Нека x е произволно. Ако $x \in A$, то
$ \exists{y} (y \in \{ A, \overline{A}\} \land  x \in y)$ от където $x in V$. Ако пък $x \notin A$, то $x \in \overline{A}$
и отново $ \exists{y} (y \in \{ A, \overline{A}\} \land x \in y)$, т.е.
$x \in V$. След $ \forall{x} (x \in V)$, противоречие!
\bigbreak

$ \overline{0} = \emptyset $\\
$ \overline{1} = \{ \overline{0}\}$\\
$ \overline{2} = \overline{1} \cup \{ \overline{1}\} = \{ \overline{0}, \overline{1}\}$\\
...\\
$ \overline{n + 1} = \overline{n} \cup \{ \overline{n}\}$ (n + 1 елемента)
\bigbreak

\fbox{Аксиома за степенното множество} За всяко множество A съществува множество B,\\
измежду чиито елементи са всички подмножества на А.\\
$ \forall{A} \exists{B} \forall{x} (x \subseteq A \Rightarrow x \in B)$
\bigbreak

\fbox{Тв} За всяко множество A същ. единствено множество B, т.ч.
$ \forall{x} (x \in B \Leftrightarrow x \subseteq A)$
\bigbreak

\underline{Док за същ.:} Некеа C е т.ч. $ \forall{x} (x \subseteq A \Rightarrow x \in C)$.\\
Нека $ B = \{x\ |\  x \in C \land x \subseteq A\}$. Нека $x \in B$. След $ x \in C \land x \subseteq A$, от където
$x \subseteq A$. След. $x \in C$ , от където $x \in C \land x \subseteq A$, т.е. $x \in B$\\
Заб: $x \in C \land x \subseteq A \Leftrightarrow x \subseteq A$, защото $ x \subseteq A \Rightarrow x \in C$
\bigbreak

\underline{Док за единственост:} Взимаме $B_1, B_2$ и $ \forall{x} (x \in B_i \Leftrightarrow x \subseteq A)$\\
$ x \in B_1 \Leftrightarrow x \subseteq A \Leftrightarrow x \in B_2$, т.е. $B_1 = B_2$.\\
Такова множество B съществува и е единствено и ще означаваме с $ \mathcal{P}(A) = \{x\ |\ x \subseteq A\}$
\bigbreak

\underline{Какво можем да изведем от тук?}
\begin{itemize}
  \item $ \emptyset \in \mathcal{P}(A) $ за всяко A
  \item $ A \in \mathcal{P}(A)$, за всяко A
  \item $ A \in \mathcal{P}(A)$, за всяко A
  \item $ A \subseteq B \implies \mathcal{P}(A) \subseteq \mathcal{P}(B) $ - монотонност
  \item Можем ли да твърдим монотонността в обратната посока? Да!
  \item Възможно ли е $ \mathcal{P}(A) \subseteq A$? Не! (дори и за празното).
    Това е същото като $ \mathcal{P}(A) \in  \mathcal{P}(A)$, но това все още не можем да докажем.
\end{itemize}
\bigbreak

Но можем да докажем следното:\\
\fbox{Тв} Не същ. множество A, т.ч. $ \mathcal{P}(A) \subseteq A$
\bigbreak

\underline{Док:} Допускаме противното и нека A е такова множество, че $ \mathcal{P}(A) \subseteq A$.\\
Нека $ \mathcal{R}_A = \{x\ |\ x \in A \land x \notin x\}$.
Според аксиомата схема за отделяне $ \mathcal{R}_A$ е множество.
Освен това, $ \mathcal{R}_A \subseteq A$. След $ \mathcal{R}_A \in \mathcal{P}(A) $ и по допускане
$ \mathcal{P}(A) \subseteq A$, от където $ \mathcal{R}_A \in A$.\\
Но $ \mathcal{R}_A \in A \Leftrightarrow \mathcal{R}_A \in A \land \mathcal{R}_A \notin \mathcal{R}_A \Leftrightarrow 
\mathcal{R}_A \notin \mathcal{R}_A$. Противоречие! След. $ \lnot \exists{A} ( \mathcal{P}(A) \subseteq A)$
\bigbreak

\fbox{Опр} Казваме, че множеството z е транзитивно, ако $z \subseteq \mathcal{P}(z) $. (ще бележим с $trans(z)$)\\
Тоест z е транзитивно $ \Leftrightarrow \forall{y} (y \in z \Rightarrow y \subseteq z) \Leftrightarrow \forall{x} \forall{y} (x \in y \land y \in z \Rightarrow x \in z)$\\
$\bigcup{z} \subseteq z$
\bigbreak

\fbox{Тв} Нека x е множество. Тогава:
\begin{enumerate}
  \item $trans(x) \Rightarrow trans(\bigcup{x})$
  \item $ \forall{y} (y \in x \Rightarrow trans(y)) \Rightarrow trans(\bigcup{x})$
  \item $ \forall{y} (y \in x \Rightarrow trans(y)) \Rightarrow trans(\bigcap{x})$
  \item $trans(x) \Rightarrow trans( \mathcal{P}(x) )$
  \item $trans(x) \Rightarrow trans(x \cup \{ x\})$
\end{enumerate}
Заб: $S(x) = x \cup \{ x\}$ е наследник на x
\bigbreak

\underline{Док 1:} Нека x е транз. Нека $y \in \bigcup{x}$. Следователно $ \exists{z} (y \in z \land z \in x)$. Нека $z_0$
е свидетел за това: $y \in z_0, z_0 \in x$.
Но $trans(x)$, от където $y \in x$. От $y \in x$, винаги е вярно че $y \subseteq \bigcup{x} $.
Тогава $y \subseteq \bigcup{x} $. След $ \bigcup{x} $ е транзитивно.
\bigbreak

\underline{Док 2:} Нека вс. ел. на x е транзитивно множество. Нека $y \in \bigcup{x} $. Нека $z$ е т.ч. $y \in z \land z \in x$.
Но z е транзитивно ($z \in x$) значи $y \subseteq z$. Понеже $z \in x$, то $z \in \bigcup{x} $.
Така $y \subseteq z \land z \subseteq \bigcup{x} $, от където $y \subseteq \bigcup{x} $. Т.е. $trans( \bigcup{x} ) $
\bigbreak

\underline{Док 3:} Нека x е множество от транзитивни множества.\\
Заб: Трябва да внимаваме, защото $ \forall{y} (y \in \emptyset \Rightarrow trans(y) )$\\
Ако $x = \emptyset $, то $ \bigcup{x} = \bigcup{ \emptyset } = \emptyset $\\
Нека сега $x \neq \emptyset $. Нека $y \in \bigcap{x}$. Тогава $ \forall{z} (z \in x \Rightarrow y \in z)$.
Понеже $ \forall{z} (z \in x \Rightarrow trans(z) )$, то $ \forall{z} (z \in x \Rightarrow y \subseteq z)$.
Така $y$ съдържа елементи, които са общи за всички елементи на $x$. Тогава $y \subseteq \bigcap{x} $.
Следователно $ trans( \bigcap{x} )$.
\bigbreak

\underline{Док 4:} Нека $trans(x)$.\\
Можем да използваме че $ \bigcup{z} \subseteq z$
и можем да докажем следното $ \bigcup{ \mathcal{P}(x) = x \subseteq \mathcal{P}(x) } $
\bigbreak

Друг подход:\\
$trans(x) \implies x \subseteq \mathcal{P}(x) \implies \mathcal{P}(x) \subseteq \mathcal{P}( \mathcal{P}(x) ) \implies trans( \mathcal{P}(x) )$
\bigbreak

\underline{Док 5:} Нека $trans(x)$. Нека $y \in S(x) = x \cup \{ x\}$.
Ако $y \in x$, то понеже $trans(x)$ имаме че $y \subseteq x$. Но $x \subseteq S(x) = x \cup \{ x\}$.
Така $y \subseteq S(x)$. Ако $y \in \{ x\}$, то $y = x \subseteq S(x)$.\\
$ \forall{y} (y \in S(x) \Rightarrow y \subseteq S(x))$. Така $trans(S(x))$
\bigbreak

Въвеждаме още съкратен синтаксис (синтактична захар) за $ \phi(x)$ и $A$ - множество:
\begin{itemize}
  \item $ (\exists{x \in A}) ( \phi(x)) \leftrightharpoons \exists{x} (x \in A \land \phi(x) )$
  \item $ (\forall{x \in A}) ( \phi(x)) \leftrightharpoons \forall{x} (x \in A \Rightarrow \phi(x) )$
  \item $ \exists{!x} ( \phi(x) ) \leftrightharpoons \exists{x} ( \phi(x) \land \forall{y} ( \phi(y) \Rightarrow x = y))$
\end{itemize}

\clearpage
\begin{center}
  \underline{\huge\normalfont Лекция 3}
\end{center}
\bigbreak

\fbox{def} Декартово произведение\\
$A \times B = \{<a,b>\ |\ a \in A \land b \in B\}$.
Тук $ \phi(x) \leftrightharpoons \exists{a} \exists{b} ( x = <a,b> \land a \in A \land b \in B)$\\
и $ x \in A \times B \Leftrightarrow \phi(x) $
\bigbreak

\underline{Наблюдение:} $ <a,b>, a \in A$ и $b \in B$\\
$ \{ a\} \subseteq  A \subseteq A \cup B$, $ \{ a,b\} \subseteq A \cup B$\\
$ \{ a\}, \{ a, b\} \in \mathcal{P}(A \cup B) $\\
$ \{ \{ a\}, \{ a,b\}\} \subseteq \mathcal{P}(A \cup B) $ \\
$ <a,b> \subseteq \mathcal{P}(A \cup B) $ \\
$ <a,b> \in \mathcal{P}( \mathcal{P} (A \cup B)) $ \\

\bigbreak
\fbox{Тв} За вс. 2 мн-ва $A$ и $B$, същ. единствено мн-во $C$, такова че:\\
$ \forall{u} (u \in C \Leftrightarrow \exists{a} \exists{b} (a \in A \land b \in B \land u = <a,b>))$
\bigbreak

\underline{Док за единственост:} за домашна.
\bigbreak

\underline{Док за съществуване:} Нека $C = \{u\ |\ u \in \mathcal{P}( \mathcal{P}(A \cup B) ) \land \phi(u)\} $.\\
Имаме че $ \forall{u} ( \phi(u) \Rightarrow u \in \mathcal{P}( \mathcal{P}(A \cup B) )$, от където
$ \forall{u} (u \in C \Leftrightarrow \phi(u) )$.
Това единствено множество ще бележим с $A \times B$ и ще наричаме декартово произведение на A и B.
\bigbreak

\fbox{Тв} За всеки $A, B, C$ - множества, е в сила че:
\begin{enumerate}
  \item Ax$\emptyset = \emptyset $
  \item $ \exists{A} \exists{B} (AxB = BxA)$, т.е. операцията не е комутативна
  \item $(AxB)xC = Ax(BxC)$ ? Не е асоциативна!
  \item $A \times (B \cup C) = (A \times B) \cup (A \times C)$
  \item $A \times (B \cap C) = (A \times B) \cap (A \times C)$
  \item $B \times (\bigcup A) = \bigcup \{ B \times x\ |\ x \in A\}$ - като за начоло се питаме синтаксиса коректен ли е?
    Тоест това от десния край е мн-во ли е?
\end{enumerate}
\bigbreak

\underline{Док 5:}\\
$(\subseteq)$ Нека $x \in A \times (B \cap C )$. Нека $a \in A, b \in B \cap C$ са т.ч.
$x = <a,b>$. Но $b \in B, b \in C$,\\ от където $<a,b> \in A \times B$ и $<a,b> \in A \times C$.
Така $x \in (A \times B) \cap (A \times C)$.
\bigbreak

$(\supseteq)$ Нека $x \in (A \times  B) \cap (A \times C)$. Тогава $x \in A \times B$ и $x \in A \times C$. Нека
$a \in A, b \in B$,\\
т.ч. $x = <a,b>$. Нека $a' \in A $ и $ c \in C$ са такива че $x = <a',c>$>.\\
Понеже $<a,b> = x = <a',c>$, то $a = a'$ и $b = c$.
Следователно $b \in B \cap C$, от където $x = <a,b> \in A \times (B \cap C)$.
\bigbreak

\underline{Док 6:} Първо да докажем че операцията е коректна.\\
$B \times x, x \in A \implies x \subseteq \bigcup{A} \implies B \times x \subseteq B \times (\bigcup A) \implies B \times x \in \mathcal{P}(B \times (\bigcup A)) $.\\
Тук $M \leftrightharpoons \mathcal{P}(B \times (\bigcup A))$, което ще е резултат от отделянето.
\bigbreak

\fbox{Лема} Съществува единствено мн-во $ \forall{u} (u \in C \Leftrightarrow ( \exists{x} \in A)(u = B \times x)) $
\bigbreak

\underline{Док:} Единственост - от аксиомата за обемност.
\bigbreak
(съществуване): Нека $C = \{u\ |\ u \in \mathcal{P}(B \times \bigcup A) \land ( \exists{x \in A})(u = B \times x) \}$.\\
$u \in C \implies u \in \mathcal{P}(B \times  \bigcup{A} ) \land \phi(u) \implies \phi(u)  $. Сега от $ \phi(u) \Rightarrow u \in \mathcal{P}(B \times  \bigcup{A} )$ следва ...
\bigbreak
$( \subseteq)$ Нека $ u \in B \times ( \bigcup{A} )$ е произволно. Нека $<b,c> = u$ като $b \in B$ и $c \in \bigcup{A} $.
Нека $a \in A$ е т.ч. $c \in a$. Тогава $u = <b,c> \in B \times a, a \in A$. Но $B \times a \in \{B \times x\ |\ x \in A\}$, от където $u \in \bigcup{ \{B \times x\ |\ x \in A\}} $
\bigbreak

$(\supseteq)$ Нека $u \in \bigcup{ \{B \times x\ |\ x \in A\}} $. Нека $a \in A$ е т.ч. $u \in B \times a$.
Нека $b \in B, c \in a$ са т.ч. $u = <b,c>$. Но $a \in A \implies a \subseteq \bigcup{A}$, така $c \in \bigcup{A} $.
Тогава $u = <b,c> \in B \times ( \bigcup{A} )$.
\bigbreak

\hrule
\bigbreak
\underline{Сега от Тинко:}\\
Множествата са естествени числа - $\mathbb{N}$,\\
т.е. един обект е множество $\Longleftrightarrow $ този обект е естествено число.
\bigbreak

Нека $ x $ и $ y $ са множества, $x = y \Longleftrightarrow x = y$ като естествени числа.
\bigbreak

Сега ще дефинираме принадлежност.\\
Нека $n>0$, тогава $n = (1 b_{k-1} ... b_1 b_0) = 1.2^k + ... + b_1.2^1 + b_0.2^0$\\
Нека $ x $ и $ y $ са множества. Казваме че $y \in x$ ако $b_{y-1} = 1$ в двоичното представяне на $ x $.
\bigbreak

Вижда се че логическите аксиоми са в сила - еквивалентност на равенството и конгруентност.
\bigbreak

Какво означава аксиомата за екстенсионалност
$ \forall{x} \forall{y} ( \forall{z} (z \in x \Leftrightarrow z \in y) \Rightarrow x = y)$?\\
Ами $ x $ и $ y $ имат еднакви двоични представяния, т.е. те са равни.
\bigbreak

\underline{Аксиома за чифта:} Нека $a$ и $b$ са множества:
\begin{enumerate}
  \item $a = b$, тогава $x = 2^a$
  \item $a \neq b$, тогава $x = 2^a + 2^b$
\end{enumerate}
\bigbreak

\underline{Схема за отделяне:} Нека $ \phi(x) $ е ТМ свойство.\\
Нека $ A $ е съвкупността на естествените числа $ x $,
за които $ \phi(x) $ е вярно. Нека B е множество.\\
Сега се чудим дали $ \exists{C} \forall{x} (x \in C \Leftrightarrow x \in B \land \phi(x) )$ е изпълнено.\\
Ами това са тези битове $b$ на $ B $, за които е вярно свойството $ \phi(b) $. Съответно в двоичния запис на $ C $ само на съответните позиции на тези $b$-та има 1, на всички останали има 0.
\bigbreak

\fbox{Аксиома за безкрайност}
\bigbreak
\underline{Форма на Цермело:} $ \exists{A} ( \emptyset \in A \land \forall{x} (x \in A \Rightarrow \{ x\} \in A))$\\
Нека $A_0$ е множество със свойството $ \emptyset \in  A_0 \land \forall{x} (x \in A_0 \Rightarrow \{ x\} \in A_0)$.
$ \emptyset \in A_0 \Rightarrow \{ \emptyset \} \in A_0$, така $ \{ \emptyset \} \in A_0$ и т.н. показваме за произволен брой влагания на $ \emptyset $.
\bigbreak

\underline{Форма на Фон Нойман:} $ \exists{A} ( \emptyset \in A \land \forall{x} (x \in A \Rightarrow x \cup \{ x\} \in A ))$\\
$A_0$: $ \emptyset \in  A_0, \{ \emptyset \} \in  A_0, \{ \emptyset , \{ \emptyset \}\} \in  A_0$ и т.н.
Ние ще ползваме тази дефиниция когато говорим за естествени числа, където $0 \leftrightharpoons \emptyset $.
\bigbreak

\fbox{Аксиома за регулярност/фундираност}
$ \forall{x} (x \neq \emptyset \Rightarrow \exists{y} (y \in x \land y \cap x = \emptyset ))$\\
(Формулирана от Мириманов през 1917г и от Фон Нойман през 1925г)
\bigbreak

\fbox{T}
\begin{enumerate}
  \item $ \lnot \exists{x} (x \in x)$
  \item $ \lnot \exists{x} \exists{y} (x \in y \land y \in x)$
  \item $ \lnot \exists{x} \exists{y} \exists{z} (x \in y \land y \in z \land z \in x)$
  \item Не съществува редица от мн-ва $x_0, x_1, x_2, ..., x_n, x_{n+1}, ...$, такива че$x_0 \in x_1, x_1 \in x_2, ...$
\end{enumerate}
\bigbreak

\underline{Док 1:} Да допусканем, че $ \exists{x} (x \in x)$. Нека $x_0$ е свидетел за това съществуване, т.е. нека $x_0$ е
мн-во със свойството $x_0 \in x_0$. Нека $x_1 = \{ x_0\}$, т.е. $x_0 \in x_1$. Значи $x_1 \neq \emptyset$, следователно
$ \exists{y} (y \in x_1 \land y \cap x_1 = \emptyset)$. Нека $y_0$ е свидетел за това съществуване, т.е. $y_0 \in x_1 \land y_0 \cap x_1 = \emptyset$.\\
Така $y_0 \in x_1$, но $x_1 = \{ x_0\}$, следователно $y_0 = x_0$ и така $x_0 \in x_0$.\\
Следователно $x_0 \in y_0$, $x_0 \in \{ x_0\}, \{ x_0 = x_1\}$. Така $x_0 \in y_0$ и $x_0 \in x_1$. Значи $x_0 \in y_0 \cap x_1$.\\
Това е абсурд, понеже $y_0 \cap x_1 = \emptyset$.
\bigbreak

\underline{Док 2:} Да доп. че $ \exists{x} \exists{y} (x \in y \land y \in x)$. Нека $x_0$ и $y_0$ са мн-ва, т.ч.
$x_0 \in y_0 \land  y_0 \in x_0$. Нека $x_1 = \{ x_0, y_0\}$. Така $x_1 \neq \emptyset$. От $x_1 \neq \emptyset \implies \exists{y} (y \in x_1 \land  y \cap x_1 = \emptyset)$.\\
Следователно $ \exists{y} (y \in x_1 \land  y \cap x_1 = \emptyset)$. Нека $y_1$ е такова мн-во, че $y_1 \in x_1 \land  y_1 \cap x_1 = \emptyset$. $y_1 \in x_1, x_1 = \{ x_0, y_0\}$. Следователно $y_1 = x_0 \lor y_1 = y_0$. Да разгледаме случаите:
\begin{enumerate}
  \item $y_1 = x_0$. Разглеждаме $y_0$. Знаем че $y_0 \in x_0$ и $x_0 \in y_0$. Така $y_0 \in y_1$, но $y_0 \in x_1$ защото $x_1 = \{ x_0, y_0 \} \implies y_0 \in y_1 \cap x_1 \implies$ противоречие $y_1 \cap x_1 = \emptyset$
  \item $y_1 = y_0$. $x_0 \in y_0$, следователно $x_0 \in y_1$. Така ?...?
  \item Сами! Hint: Допускаме че $x_0 \in y_0 \land  y_0 \in z_0 \land  z_0 \in x_0$ и $x_1 \leftrightharpoons \{ x_0, y_0, z_0\}$
\end{enumerate}

\fbox{Аксиомна схема за замяната} (С тази аксиома вече имаме аксиомната схема $ \mathcal{ZF} $) \\
Имаме един детерминистичен преобразувател (на интуитивно ниво функция) - $ \phi(x,y,\overline{u})$,\\
в който можем да фиксираме $\overline{u}$ и за дадено $ x $ то ни връща $ y $.\\
Аксиомната схема твърди, че за такова $\phi$ с дефиниционна област $ A $, има съответен образ на $\phi$. Френкел забелязва че ако разгледаме $ \mathbb{N}, \mathcal{P}( \mathbb{N}), ..., \mathcal{P}^n( \mathbb{N})  $, то не можем да гарантираме че това последното $ \mathcal{P}(N)^n $ съществува.
\bigbreak

\underline{Схемата:} Нека
$\forall{u_1}... \forall{u_n} ((\forall{x} \forall{y_1} \forall{y_2}  (\phi(x,y_1,\overline{u})\land \phi(x,y_2,\overline{u})) \Rightarrow y_1 = y_2) \Rightarrow \forall{A} \exists{B} \forall{z} (z \in B \Leftrightarrow \exists{x} (x \in A \land \phi(x,z,\overline{u}))))$
\bigbreak

Разлглеждаме: $ \mathcal{P}(\emptyset) = \{ \emptyset\}, A \leftrightharpoons \mathcal{P}(\{\emptyset\}) = \{ \emptyset, \{ \emptyset\}\} $\\
$ \phi(x,y) \leftrightharpoons (x = \emptyset \land y = a) \lor (x = \{ \emptyset\} \land y = b) )$\\
$ \forall{x} \forall{y_1} \forall{y_2} ( \phi(x,y_1) \land \phi(x,y_2) \Rightarrow  y_1 = y_2)$\\
$ \exists{B} \forall{z} (z \in B \Leftrightarrow \exists{x} (x \in A \land \phi(x,z)))$\\
\bigbreak

\fbox{Аксиома за избора ($ \mathcal{AC} $)} Нека имам някакво разделяне(разбиване) на множеството A и взема по един елемент от всяка част.
$ \forall{z} (z \in A \Rightarrow  \bigcap z$ е синглетон ($ \exists{u} (z \cap c = \{ u\})$))
\bigbreak
С помощта на аксиомата за избора се доказва че всяко множество може да бъде добре наредено (contraversial).
$ \forall{x} (x \in A \Rightarrow \emptyset) \Rightarrow \exists{f} (Func(f))$\\
$ Fom(f) = A \land \forall{x} (x \in A \Rightarrow f(x) \in x)$\\
Това поражда и парадокса на Банарх Тарски: Взимаме кълбо B с $r = 1$,
значи може да разделим $B = B_1 \cup B_2 \cup ... \cup B_7$. След което можем да вземем
$B_1 \cup B_2 \cup B_3$ с $r = 1$ и $B_4 \cup B_5 \cup ... \cup B_17$ с $r = 1$ - абсурд!
Аксиомата за избора не е конструктивна!
\bigbreak

\underline{Аксиомата:} Аксиома на мултипликативност - форма на Ръсел, защото още не сме въвели понятието за функция\\
$ \forall{A} ( \forall{x} (x \in A \Rightarrow x \neq \emptyset) \land \forall{x} \forall{y} (x \in A \land y \in A \land x \neq y \Rightarrow x \cap y = \emptyset) \Rightarrow \exists{C} \forall{x} (x \in A \Rightarrow  \exists{u} (x \cap C = \{ u\})))$
\bigbreak

\underline{Заб:} Ако $A$ е крайно - всичко е точно. Но ако $A$ е безкрайно вече е различно.
\bigbreak

$ f: A \rightarrow B, A \twoheadrightarrow B$ (сюрекция), то можем да ограничим домейна за да получим биекция.
Тоест съществува $A_0 \subseteq A : f \upharpoonright A_0$ е биекция м-ду $A_0$ и $B$.

\clearpage
\begin{center}
  \underline{\huge\normalfont Лекция 4}
\end{center}
\bigbreak

\fbox{(Бинарни) Релации} Това са множества (обекти от света ни).
\bigbreak
\underline{Пример:} $P_1(A,l) \leftrightharpoons$ точката $ A $ лежи на правате $l$.\\
Обаче може да имаме различни свойства, които описват еднакви релации (множества).\\
Ако $P_2(A,l) \leftrightharpoons$ правата $l$ минава през точката $A$,\\
то $R_1 = \{<A,l>\ |\ P_1(A,l)\}$ и $R_2 = \{<A,l>\ |\ P_2(A,l)\}$ са равни.
За нас релация е просто множество от наредени двойки $R \subseteq A \times B$
\bigbreak

\fbox{Опр} Бинарна релация е множество, чийто елементи са наредени двойки.\\
$Rel(R) \leftrightharpoons \forall{z}(z \in R \Rightarrow \exists{x} \exists{y}(z = <x,y>))$
\bigbreak

\underline{Примери:}
\begin{enumerate}
  \item $ \emptyset$ - никъде недефинираната релация
  \item $A$ е множество, $A \times A$ е релация (пълна релация над $A$)
  \item $A$ е мн-во, $id_A = \{<x,x>\ |\ x \in A\}$ е идентитет на $A$.\\
    \underline{Заб:} $T = \{<x,x>\ |\ x = x\}$ не е множество (поражда парадокса на Ръсел)
\end{enumerate}
\bigbreak

\fbox{def} Нека $R$ е релация.\\
Дефиниционна област на $R$ наричаме: $Dom(R) \leftrightharpoons \{x\ |\ \exists{y}(<x,y> \in R)\}$\\
Област на стойностите на $R$ наричаме: $Rng(R) \leftrightharpoons \{y\ |\ \exists{x}(<x,y> \in R)\}$
\bigbreak

\fbox{Тв} За всяка релация $R$, $Dom(R)$ и $Rng(R)$ са множества.
\bigbreak
\underline{Док:} $x \in Dom(R) \implies \exists{y}(<x,y> \in R) \implies \exists{y}( \{ x\} \in <x,y> \in R) \implies \{ x\} \in \bigcup R \implies Dom(R) \subseteq \bigcup \bigcup R, Dom(R)$ е определима съвкупност (клас).\\
$\bigcup \bigcup R$ е множество $ \implies Dom(R)$ е множество.\\
Аналогично получаваме $y \in Rng(R) \Rightarrow y \in \bigcup \bigcup R \implies Rng(R) \subseteq \bigcup \bigcup R$ - множество.
\bigbreak

\fbox{Тв} $Rel(R) \Rightarrow R \subseteq Dom(R) \times Rng(R)$
\bigbreak
\underline{Док:} Нека $z \in R$. Тогава $z$ е наредена двойка. Нека $x$ и $y$ са т.ч. $z = <x,y>$.\\
Тогава $x \in Dom(R)$ и $y \in Rng(R)$. Следователно $z = <x,y> \in Dom(R) \times Rng(R)$
\bigbreak

\fbox{Операции върху релации} $R,S$ - релации, то $\implies R \cup S, R \cap S, R \setminus S$ са релации\\
$R^{-1} \leftrightharpoons \{<x,y>\ |\ <y,x> \in R\}$ е съвкупност от наредени двойки. Обаче множество ли е?\\
$R^{-1} = \{<x,y>\ |\ <y,x> \in R\} = \{u\ |\ \exists{x}\exists{y}(u = <x,y> \land <y,x> \in R)\}
= \{u\ |\ u \in Rng(R) \times Dom(R) \land \exists{x,y}(u = <x,y> \land <y,x> \in R)\}$
\bigbreak

\fbox{def} Операция - композиция на релации. $(f \circ g)(x) = g(f(x))$
\bigbreak

\fbox{Опр} Композицията на релациите R и S наричаме мн-вото:\\
$R \circ S \leftrightharpoons \{<x,y>\ |\ \exists{z}(<x,z> \in R \land <z,y> \in S)\}
= \{u\ |\ (\exists{x,y,z})(u = <x,y> \land <x,z> \in R \land <z,y> \in S)\}
= \{u\ |\ u \in Dom(R) \times Rng(S) \land (\exists{x,y,z})(u = <x,y> \land <x,z> \in R \land <z,y> \in S)\}$
\bigbreak

\fbox{Св} Нека $R, S_1, S_2$ са релации. Тогава са изпълнени:
\begin{enumerate}
  \item $R\circ (S_1\circ S_2) = (R\circ S_1)\circ S_2$ - асоциативност
  \item $(S_1 \cup S_2)\circ R = (S_1\circ R) \cup (S_2\circ R)$\\
    $R \circ (S_1 \cup S_2) = (R \circ S_1) \cup (R \circ S_2)$
  \item $R \circ (S_1 \cap S_2) \subseteq (R \circ S_1) \cap (R \circ S_2)$, обратното включване не винаги е вярно.
  \item $R \circ (S_1 \setminus S_2) \supseteq (R \circ S_1) \cap (R \circ S_2)$
  \item $(S_1 \circ S_2)^{-1} = S_2^{-1} \circ S_2^{-1}$
\end{enumerate}
\bigbreak

\underline{Док 3:} Нека $u \in R \circ (S_1 \cap S_2)$. Нека $x,y,z$,
т.ч. $u = <x,y>, <x,z> \in R$ и $<z,y> \in S_1 \cap S_2$. Тогава $<z,y> \in S_1$ и $<z,y> \in S_2$.
След. $<x,y> \in R \circ S_1$ и $<x,y> \in R \circ S_2$. Така $u = <x,y> \in (R \circ S_1) \cap (R \circ S_2)$.
\bigbreak
\hrule
\bigbreak
$R = \{ <x,z>, <x,t>\}, z \neq t$\\
$S_1 = \{ <z,y>\}$\\
$S_2 = \{ <t,y>\}$\\
$S_1 \cap S_2 = \emptyset, R \circ (S_1 \cap S_2) = R \circ \emptyset = \emptyset$\\
$R \circ S_1 = {<x,y>}$\\
$R \circ S_2 = {<x,y>}$\\
$\implies (R \circ S_1) \cap (R \circ S_2) = \{ <x,y>\}$
\bigbreak

\underline{Док 4:} Нека $u \in (R \circ S_1) \setminus (R \circ S_2)$. Така $u \in R \circ S_1$ и $u \notin R \circ S_2$.
Нека $x,y,z$ са такива че $<x,z> \in R$ и $<z,y> \in S_1$. Понеже $u \notin R \circ S_2$,
то $\forall{t}(<x,t> \in R \Rightarrow <t,y> \notin S_2)$. Но $<x,z> \in R$, след. $<z,y> \notin S_2$.
Обаче $<z,y> \in S_1$, от където $<z,y> \in S_1 \setminus S_2$. От $<x,z> \in R$,
следва че $u = <x,y> \in R \circ (S_1 \setminus S_2)$\\
Обратното не е винаги вярно!
\bigbreak

\underline{Заб:} $ (\circ) $ не е комутативна!
\bigbreak

\fbox{Опр} Нека $Rel(R)$ и $A \subseteq Dom(R)$. Образ на $A$ при $R$ наричаме множеството:\\
$R[A] = \{y\ |\ \exists{(x \in A)}(<x,y> \in R)\} \subseteq Rng(R)$
\bigbreak

\fbox{Опр} Нека $Rel(R)$ и $B \subseteq Rng(R)$. Праобраз на $B$ при $R$ наричаме множеството:\\
$R^{-1}[B ]= \{x\ |\ \exists{(y \in B)}(<x,y> \in R)\} \subseteq Dom(R)$
\bigbreak

\fbox{Тв (за коректност)} Нека $R$ е релация и $B \subseteq Rng(R)$. Тогава $(R^{-1})[B] = R^{-1}[B] $, където
$(R^{-1})[B]$ е образ на $B$ ри $R^{-1}$, а $R^{-1}[B]$ е праобраз на $B$ при $R$.
\bigbreak

\underline{Док:} За вс. $x$ е в сила че $x \in (R^{-1}[B]) \Longleftrightarrow \exists{y}(<y,x> \in R^{-1} \land y \in B)
\Longleftrightarrow \exists{y}(y \in B \land  <x,y> \in (R^{-1})^{-1}) \Longleftrightarrow \exists{y}(y \in B \land <x,y> \in R) \Longleftrightarrow x \in (R)^{-1}[B]$
\bigbreak

\fbox{Тв} Нека $\forall{x}(x \in X \Rightarrow x \subseteq Dom(R))$. Тогава $R[\bigcup X] = \bigcup \{R[x]\ |\ x \in X\}$.
Тук $Rel(R)$ и $X$ е множество.
Това е коректно защото $(\forall{x \in X})x \subseteq Dom(R)) \implies \bigcup X \subseteq Dom(R)$.
$a \in \bigcup X \implies \exists{x}(x \in X \land  a \in x) \implies a \in Dom(R)$. Сега това множество ли е?\\
Нека $x \in X \implies x \subseteq Dom(R) \implies R[x] \subseteq R[Dom(R)]$.
Тогава ако $A \subseteq A_1 \subseteq Dom(R) \implies R[A] \subseteq R[A_1]$ и съответно
$B \subseteq B_1 \subseteq Rng(R) \implies R^{-1}[A] \subseteq R^{-1}[B_1]$.
Значи това е определима съвкупност $ \{R[x]\ |\ x \in X\} \subseteq \mathcal{P}(Rng(R))$.
Всичко е коректно, сега доказателството.
\bigbreak
\underline{Док:} Нека $b \in R[\bigcup X]$. Нека $a \in \bigcup X$ е т.ч. $<a,b> \in R$. Нека $x_0 \in X$ е такъв че $a \in x_0$.
Тогава $b \in R[x_0]$. Следователно $b \in \bigcup \{R[x]\ |\ x \in X\}$
\bigbreak
Сега обратното включване. Нека $b \in \bigcup \{R[x]\ |\ x \in X\}$. Нека $x_0 \in X$ е т.ч. $b \in R[x_0]$.
Но $x_0 \subseteq \bigcup X$. Пак от монотонността следва че $b \in R[x_0] \subseteq R[\bigcup X]$.
\bigbreak

\fbox{Тв} Нека $Rel(R)$ и $X$ е мн-во за което е изп. че $\forall{x}(x \in X \Rightarrow x \subseteq Dom(R))$.
Тогава $R[\cap X] \subseteq \cap \{R[x]\ |\ x \in X\}$, като не винаги е в сила обратното включване.
Ако допълнително $(\forall{y}Rng(R))(\exists{!x \in Dom(R))(<x,y> \in R)})$ (нещо като инективност), то тогава \\
$R[\cap X] = \cap \{R[x]\ |\ x \in X\}$.
\bigbreak
\underline{Док:} Нека $b \in R[\cap X]$. Нека $a \in \cap X $ е такова че $<a,b> \in R$.
Следователно за всяко $x \in X, a \in x$. Следователно за вскяо $x \in X, b \in R[x]$.
Така $b$ принадлежи на всички елементи на $\{R[x]\ |\ x \in X\}$, значи $b \in \cap \{R[x]\ |\ x \in X\}$.
\bigbreak
\underline{Пример:} $X = \{ \{ a_1\}, {a_2}\}$ и $a_1 \neq a_2, R = \{<a_1, b_1>, <a_2, b_2> \}$\\
$\cap X = \{a_1\} \cap \{ a_2\} = \emptyset, R[\cap X] = \emptyset$\\
$R[ \{ a_1\} ] = \{y\ |\ (\exists{x \in \{ a_1\}})(<x,y> \in R)\} = \{y\ |\ <a_1,y> \in R\} = \{ b\}$.\\
Значи $R[ \{ a_2\}] = \{ b\}, \{R[x]\ |\ x \in X \} = \{ \{ b\}\}$.\\
$\cap \{ \{ b\}\} = \{ b\}$, $A = \{ a\}, a = \{ b\}$,\\
$x \in \cap A \Leftrightarrow \forall{a \in A}(x \in a)$

\bigbreak
Нека $(\forall{y \in Rng(R)})(\exists{!x \in Dom(R))(<x,y> \in R)}$. Нека $b \in \cap \{R[x]\ |\ x \in X\}$.\\
Следователно за всяко $x \in X, b \in R[x]$, т.е. за всяко $x \in X$ същ $a \in x$,
т.ч. $<a,b> \in R$.
\bigbreak
\underline{$b \in Rng(R)$:} $x \neq \emptyset$. Нека $x_0 \in X$. Тогава $b \in R[x_0]$. След $b \in Rng(R)$.
Нека $a_0 \in x_0$ е т.ч. $<a_0,b> \in R$. Нека сега $x \in X$ е произволно и $a \in x$ е т.ч. $<a,b> \in R$.
Но $<a_0,b> \in R$, от където $a_0 = a$. В частност $a_0 \in x$, но $x$ е произволнo. Следователно $a_0 \in \cap X$.
Но тогава $b \in R[\cap X]$, защото $<a_0,b> \in R$ и $a_0 \in \cap X$.
Така $\cap \{R[x]\ |\ x \in X\} \subseteq R[\cap X]$
\bigbreak

\bigbreak
\hrule
\begin{center}
\fbox{< Функции >}
\end{center}

\bigbreak
\fbox{Опр} Казваме че релацията $R$ е функция, \\
ако $Funct(R)$, където $Funct(R) \leftrightharpoons Rel(R) \land  \forall{x}\forall{y}\forall{y'}(<x,y> \in R \land  <x,y'> \in R \Rightarrow y = y')$
\bigbreak

\begin{enumerate}
  \item $Funct(R) \implies Rel(R)$
  \item $Funct(R), Dom(R) = A, Rng(R) \subseteq B$, то пишем $R: A \rightarrow B$
  \item $Funct(R), Dom(R) \subseteq A , Rng (R) \subseteq B$, то ще казваме че $R$ е частична функция от $A$ към $B$.
    Ще пишем $R: A \rightcircarrow B$
  \item $R: A \rightarrow B$ и $Rng(R) = B$, ще казваме че $R$ е сюрекция (епиморфизъм) на $A$ върху $B$.
    Означаваме с $R: A \twoheadrightarrow B$
  \item $R: A \rightarrow B, R$ е инекция (мономорфизъм),
    ако $\forall{x}\forall{x'}\forall{y}(x \neq x' \land <x,y> \in R \Rightarrow <x',y> \notin R)$.
    Означаваме $R: A \rightarrowtail B$
  \item $R: A \rightarrow B$ е биекция, ако $R$ е сюрекция на $A$ в-ху $B$ и $R$ е инекция.
    Означаваме $R: A \rightbijectionarrow B$
\end{enumerate}
\bigbreak

Понеже функциите са релации, директно се пренасят и понятията за образ и праобраз.\\
Ще използваме $f,g,h ...$, за да означаваме че дадена релация е функция.\\
Ако $Func(f)$, вместо $<x,y> \in f$ ще пишем $f(x) = y$
\bigbreak

\fbox{Следствие} Нека $Func(f)$ и нека $X$ и $Y$ са такива мн-ва че:\\
$(\forall{x \in X})(x \subseteq Dom(f))$ и $(\forall{y \in Y})(y \subseteq Rng(f))$.\\
Тогава $f[\bigcup X ] = \bigcup \{f[x]\ |\ x \in X\}$
и $f[\cap X ] \subseteq \cap \{f[x]\ |\ x \in X\}$ (равенство не винаги се достига).\\
Изпълнено е че $f ^{-1}[\bigcup X] = \bigcup \{f ^{-1} [x]\ |\ x \in X\}$
и $f ^{-1}[\cap X] = \cap \{f ^{-1} [x]\ |\ x \in X\}$.\\
$\forall{y \in Rng(R)\exists{!x \in Dom(R)(<x,y> \in R)}}$\\

\underline{($\Rightarrow$)} Нека $Func(f ^{-1})$. Нека $x, x',y$ са т.ч. $x \neq x'$ и $<x,y> \in f$.
Тогава $<y,x> \in f ^{-1}$. Ако доп, че $<x',y> \in f$, то $<y,x'> \in f ^{-1}$. Понеже $f ^{-1}$ e функция, то $x = x'$.
Но $f ^{-1}$ е функция, т.е. $x \neq x' \implies$ Противоречие! $\implies$ $<x',y> \notin f$ и значи $f$ е инективна.
\bigbreak
\underline{($\Leftarrow$)} Нека $f$ е инективна. Нека $x,y,y'$ са т.ч. $<x,y>, <x,y'> \in f ^{-1}$.\\
Тогава $<y,x>, <y',x> \in f$ и понеже $f$ е инективна то $y = y'$. Следователно $Func(f ^{-1})$.
\bigbreak

\fbox{Тв} Нека $f$ и $g$ са функции.\\
Тогава $f \circ g$ е функция с $Dom(f \circ g) = \{x\ |\ x \in Dom(f) \land f(x) \in Deom(g)\}$.\\
За всяко $x \in Dom(f \circ g)$ е вярно $ (f \circ g)(x) = f(g(x))$.
\bigbreak

\underline{Док:} $Rel(f \circ g)$. Нека $<x,y>, <x,y'> \in f \circ g$.
Нека $z,z'$ са т.ч. $<x,z> \in f \land <z',y> \in g$ и $<x,z'> \in f \land <z',y'> \in g$\\
$Func(f) \implies z = z' \implies <z,y>, <z,y'> \in g \implies y = y'$ (от $Func(g)$)
\bigbreak

Нека $x \in Dom(f \circ g)$. Нека $y$ е т.ч. $<x,y> \in f \circ g$. Нека $z$ е т.ч. $<x,z> \in f$ и $<z,y> \in g$.
Тогава $x \in Dom(f)$ и $z = f(x)$. Но $z \in Dom(g)$, от където $f(x) \in Dom(g)$.\\
Сега наобратно. Взимаме $x \in Dom(f)$ и $f(x) \in Dom(g)$. Тогава $<x,f(x)> \in f$ и $<f(x), g(f(x))> \in g$.
Следователно $<x,g(f(x))> \in f \circ g$. В частност получаваме че $x \in Dom(f \circ g)$ и понеже $Func(f \circ g)$,
то $(f \circ g)(x) = g(f(x))$.
\bigbreak

\fbox{Опр} Казваме, че функциите $f$ и $g$ са съвместими, ако $Func(f \cup g)$.
\bigbreak

\fbox{Опр}
$f: A \rightarrow B, A_1 \subseteq A$\\
Рестрикция на $f$ до $A_1$: $f \upharpoonright A_1 \leftrightharpoons f \cap (A_1 \times Rng(f))$
\bigbreak

\fbox{Тв} $f$ и $g$ са съвместими $\Longleftrightarrow f\upharpoonleft(Dom(f) \cap Dom(g)) = g \upharpoonright (Dom(f) \cap Dom(g))$
\bigbreak

\underline{Док:} \textit{To be continued...}\\
\bigbreak
\hrule
\bigbreak
Да уточним някакви неща: \\
$ f : A \rightarrow B, A_1 \subseteq A = Dom(f) $\\
Рестрикция на $f$ до $ A_1 $: $ f \upharpoonright A_1 = f \cap (A_1 \times Rng(f)) = \{<x,f(x)>\ |\ x \in A_1\}  $
\begin{enumerate}
  \item $ Funct(f \upharpoonright A_1) $
  \item $ f \upharpoonright A_1 \subseteq f \upharpoonright A$
  \item $ A_1 \subseteq A2 \subseteq A \Rightarrow f \upharpoonright A_1 \subseteq f \upharpoonright A_2$
\end{enumerate}
\bigbreak

\fbox{Опр} $ f $ и $ g $ са съвместими функции, ако $f \cup g$ е функция.
\bigbreak

\fbox{Тв} Функциите $f$ и $ g $ са съвместими $ \Leftrightarrow  $ $f \upharpoonright (Dom(f) \cap Dom(g)) =
g \upharpoonright (Dom(f) \cap Dom(g))$
\bigbreak

\underline{Док:} ($ \rightarrow $) Нека $Funct(f \cup g)$. Нека $ u \in f \upharpoonright (Dom(f) \cap Dom(g))$.
Тогава $u = <x,y>$ като $ x \in Dom(f) \cap Dom(g)$ и $ y = (f \upharpoonright (Dom(f) \cap Dom(g)))(x) = f(x) $.
Poneve $x \in Dom(g)$, то $ <x,g(x)> \in g $. Така $<x,f(x)>, <x,g(x)> \in f \cup g$. Понеже $Funct(f \cup g)$,
то $f(x) = y = g(x)$. След. $u = <x,y> = <x,g(x)> \in g$ и понеже $x \in Dom(f) \cap Dom(g)$,
то $u = <x,y> \in y \upharpoonright (Dom(f) \cap Dom(g))$.
\bigbreak
($\Leftarrow$) Нека $f \upharpoonright (Dom(f) \cap Dom(g)) = g \upharpoonright (Dom(f) \cap Dom(g))$.
Ясно е, че $Rel(f \cup g)$. Нека $<x,y>, <x,y'> \in f \cup g$ \\
Възможни са 3 случея:
\begin{enumerate}
  \item $<x,y>, <x,y'> \in f$. Но $Funct(f)$, от където $y = y'$.
  \item $<x,y>,<x,y'> \in g$. Подобно - получава се $y = y'$
  \item $<x,y> \in f , <x,y'> \in g$. Тогава $x \in Dom(f), x \in Dom(g)$. След $x \in Dom(f) \cap Dom(g)$.
    Така $y = f(x) = g(x) = y'$
\end{enumerate}
\bigbreak

\fbox{Тв} Нека $F$ е множество от две по две съвместими функции. Тогава $ \bigcup F $ е функция като:
$ Dom(\bigcup F) = \bigcup \{Dom(f)\ |\ f \in F\} $
$ Rng(\bigcup F) = \bigcap \{Rng(f)\ |\ f \in F\} $
\bigbreak
\underline{Док:} Ясно е, че $Rel(\bigcup F)$. Нека $<x,y> \in \bigcup F$ и $<x,y'> \in \bigcup F$.
Нека $f, f' \in F$ са такива че $<x,y> \in f$ и $<x,y'> \in f'$. Тогава $Funct(f \cup f')$,
като $<x,y>,<x,y'> \in f \cup f'$. Следователно $y = y'$. Така получаваме $Funct(\bigcup F)$.
\bigbreak
\hrule
\bigbreak

Нека $x \in Dom(\bigcup F)$. Нека $y$ е т.ч. $ <x,y> \in \bigcup F $. Нека $f_0 \in F$ е такова че $<x,y> \in f_0$.
Тогава $x \in Dom(f_0)$ и следователно $x \in \bigcup \{Dom(f)\ |\ f \in F\}$.
Нека сега $f_0 \in F$ е т.ч. $x \in Dom(f_0)$. Но $f_0 \subseteq \bigcup F$ и $ \bigcup F$ е функция, следователно
$Dom(f_0) \subseteq Dom(\bigcup F)$. Следователно $x \in Dom(\bigcup F)$

\clearpage
\begin{center}
  \underline{\huge\normalfont Лекция 5}
\end{center}
\bigbreak

\fbox{Опр}
За $f : \mathcal{P}(A) \rightarrow \mathcal{P}(A)$, ще казваме че $f$ е монотонна, ако: \\
$ (\forall{X_1 \supset A})(\forall{X_2 \subseteq A})(X_1 \subseteq X2 \rightarrow f(X_1) \subseteq f(X_2)) $
\bigbreak

\fbox{Опр}
И за монотонна $f : B \rightarrow B$, $ x $ е неподвижнда точка на $ f $, ако $f(x) = x$
\bigbreak

\fbox{Лема} (Тарски)\\
Нека $f: \mathcal{P}(A) \rightarrow \mathcal{P}(A)$ е монотонна функция. Тогава $f$ има неподвижнда точка.\\
Нещо повече, $ f$ има най-малка и най-голяма неподвижнда точка:\\
тоест съществуват $X_1,X_2 \in \mathcal{P}(A)$ т.ч.
$f(X_1) = X_1, f(X_2) = X_2$ и за всяко $X \in \mathcal{P}(A)$ с $f(X) = X$ е изпълнено, че $X_1 \subseteq X \subseteq X_2$.
\bigbreak

\underline{Док:} Нека $\Pi = \{X\ |\ X \subseteq A \land f(X) \subseteq X\}$
Понеже $A \in \Pi$, то $\Pi \neq \emptyset$. Нека $X_1 = \bigcap \Pi$.
За вс. $X \in \Pi$, $X_1 = \bigcap \Pi \subseteq X$. Понеже $f$ е монотонна,
то за вс. $X \in \Pi$, $f(X_1) \subseteq f(X) \subseteq X$. Следователно $f(X_1) \subseteq \bigcap \Pi = X_1$.
Понеже $X_1 \subseteq A$, то $X_1 \in \Pi$. Отново от монотонността на $f$ имаме, че
$f(f(X_1)) \subseteq f(X_1)$. Значи $f(X_1) \in \Pi$. Следователно $X_1 \subseteq f(X_1)$.
От тук $f(X_1) = X_1$ и така $X_1$ е неподвижнда точка на $f$. Ясно се вижда че:
$f(X) = X \implies x \in \Pi \implies X_1 = \bigcap \Pi \subseteq X \implies X_1$ е най-малката неподвижна точка на $f$\\
\underline{За най-голяма неподвижна точка - за домашна!}
\bigbreak
\hrule
\bigbreak

\begin{center}
(от Тинко)
\end{center}

\begin{center}
\fbox{< Равномощни множества. Сравняване на множества по мощност >}
\end{center}
\bigbreak

\fbox{def} Казваме че $A$ и $B$ са равномощни, ако съществува биекция на $A$ върху $B$,\\
тоест $\exists{f}(f: A \rightbijectionarrow B)$.
Означения: $|A| = |B| $ или $ \overline{\overline{A}} = \overline{\overline{B}}$\\
Съответно $\overline{\overline{A}} \neq \overline{\overline{B}} \leftrightharpoons \lnot(\overline{\overline{A}} = \overline{\overline{B}})$
\bigbreak
Казваме че мощността на $A$ не надминава мощността на $B$, ако $\exists{f}(f: A \rightarrowtail B)$.\\
Пишем $\overline{\overline{A}} \leq \overline{\overline{B}}$.
\bigbreak
Казваме че мощността на $A$ е строго по-малка от мощността на $B$, ако $\overline{\overline{A}} \leq \overline{\overline{B}} \land \overline{\overline{A}} \neq \overline{\overline{B}}$. Пишем $\overline{\overline{A}} < \overline{\overline{B}}$.

\fbox{Св}
\begin{enumerate}
  \item $\overline{\overline{A}} = \overline{\overline{A}}, Id_A: A \rightbijectionarrow A$
  \item $\overline{\overline{A}} = \overline{\overline{B}} \implies \overline{\overline{B}} = \overline{\overline{A}}$
  \item $\overline{\overline{A}} = \overline{\overline{B}} \land \overline{\overline{B}} = \overline{\overline{C}}
    \implies \overline{\overline{A}} = \overline{\overline{C}}$
\end{enumerate}

\underline{Док 2:} Нека $f_0: A \rightbijectionarrow B$ (свидетел за съществуващата биекция), тогава $f_0^{-1}: B \rightbijectionarrow A$.\\
Значи $\exists{f'}(f': B \rightbijectionarrow A)$
\bigbreak

\underline{Док 3:} $\exists{f}(A \rightbijectionarrow B)$ и $ \exists{f}(B \rightbijectionarrow C)$. Нека вземем
свидетели: $f_0: A \rightbijectionarrow B, f_1: B \rightbijectionarrow C$. \\
Нека $h = f_0 \circ f_1$, т.е. $h(x) = f_1(f_0(x))$. Вижда се че $h: A \rightbijectionarrow C$.
Следователно $\exists{f'}(f': A \rightbijectionarrow C)$
\bigbreak
\hrule
\bigbreak
Нека $A$ и $c$ са произволни множества. Тогава $\overline{\overline{A}} = \overline{\overline{A \times \{ c \} }}$
и $\overline{\overline{A}} = \overline{\overline{\{ c \} \times  A}}$\\
Дефинираме $f: f(a) = <a,c>$ за вс. $a \in A$.
$f = \{u\ |\ \exists{a}(a \in A \land u = <a,<a,c>>)\} =$ \\
$\{<a,c>\ |\ a \in A\}$, съответно тук отделяме $u \in A \times \{ c \}$. \textbf{idk???}
\bigbreak

\fbox{Тв} $A \neq \emptyset \Longleftrightarrow \lnot \exists{B}\forall{x}(x \in B \Leftrightarrow \overline{\overline{x}} = \overline{\overline{A}})$
\bigbreak
\underline{Док:} Нека $A \neq \emptyset$. $a_0 \in A$. Да доп. че $\exists{B}\forall{x}(x \in B \Leftrightarrow \overline{\overline{x}} = \overline{\overline{A}})$.
Нека $B$ е свидетел за съществуването $\forall{x}(x \in B \Leftrightarrow \overline{\overline{x}} = \overline{\overline{A}})$. И нека вземем $B_0 \leftrightharpoons \{w\ |\ w \in B \land \exists{c}(w = A \times \{ c \})\}$.
Значи $Rel(\bigcup B_0)$.\\
Нека $t$ е произволнo множеството, тогава $A \times \{ t \} \in B_0$. Значи за $<a_0,t> \in \bigcup B_0$.\\
Тогава $t \in Rng(\bigcup B_0)$. Така, $\forall{t}(t \in Rng(\bigcup B_0))$ - абсурт! (от допускането че $B_0$ съществува).
\bigbreak
Допускането че $A \neq \emptyset$ беше съществено.\\
Ако $A = \emptyset$, то $\exists{B}(x \in B \Leftrightarrow \overline{\overline{x}} = \overline{\overline{A}})$.
И единствената възможност е $B = \{ \emptyset \}$
\bigbreak
\fbox{Тв} $\forall{A}\forall{B}\exists{A'}\exists{B'}(\overline{\overline{A}} = \overline{\overline{A'}} \land \overline{\overline{B}} = \overline{\overline{B'}} \land A' \cap B' = \emptyset )$\\
Ще дефинираме $\overline{\overline{A}} + \overline{\overline{B}} \leftrightharpoons \overline{\overline{A' \cup B'}}$.
Как го постигаме?\\
Взимаме $c_1 \neq c_2$ и тогава $A' \leftrightharpoons A \times \{ c_1c1 \}$ и
$B' \leftrightharpoons B \times \{ c_2 \}$. А $A' \cap B' = \emptyset$
\bigbreak

\fbox{Св} Ако $\overline{\overline{A'}} = \overline{\overline{A''}} \land \overline{\overline{B'}} = \overline{\overline{B''}} \land  A' \cap B' = \emptyset \land  A'' \cap B'' = \emptyset \implies \overline{\overline{A' \cup B'}} = \overline{\overline{A'' \cup B''}}$
\bigbreak
\underline{Док:} Взимаме свидетели $f_1: A' \rightbijectionarrow A''$ и $ f_2: B' \rightbijectionarrow B''$, тогава
$f_1 \cup f_2$ е функция, защото $f_1$ и $f_2$ са съвместими.
Съответно $Dom(f_1 \cup f_2) = Dom(f_1) \cup Dom(f_2) = A' \cup B'$. Аналогично за за $Rng(f_1 \cup f_2)$

\fbox{def} Бихме искали да го дефинираме така
$\overline{\overline{A}}.\overline{\overline{B}} = \overline{\overline{A \times B}}$.\\
Пак ще вземем равномощни на $A$ и $B$.
$\overline{\overline{A}} = \overline{\overline{A'}} \land  \overline{\overline{B}} = \overline{\overline{B'}} \implies
\overline{\overline{A x B}} = \overline{\overline{A' \times  B'}}$.
\bigbreak

\fbox{def} Ами степенуване - $k^n$? Ако $\overline{\overline{A}} = k$ и $\overline{\overline{B}} = n$, то това ще са всички функции от $B$ в $A$. \\
Искаме да покажем $\overline{\overline{A}}^{\overline{\overline{B}}} = \overline{\overline{^B A}}$.
Тоест $B_A = \{f\ |\ f: B \rightarrow A\}$.\\
$\overline{\overline{A}} = \overline{\overline{A'}} \land \overline{\overline{B}} = \overline{\overline{B'}} \implies
\overline{\overline{A}}^{\overline{\overline{B}}} = \overline{\overline{^{B'} A'}}$\\


\end{document}
% Some commented notes
% $\mathbb{set}$
% \mathcal{letter} ръкописни
% _ and ^ for upper/lower index
